\documentclass[reqno,a4paper,12pt]{amsart}

\usepackage{amsmath,amssymb,amsthm,geometry,xcolor,soul,graphicx}
\usepackage{titlesec}
\usepackage{enumerate}
\usepackage{lipsum}
\usepackage{listings}
\RequirePackage[most]{tcolorbox}
\usepackage{float}  %可以在colorbox环境下添加table环境 此时,\begin{table}[H] 位置需要H
\usepackage{braket}
\allowdisplaybreaks[4] %align公式跨页
\usepackage{xeCJK}
\setCJKmainfont[AutoFakeBold = true]{Kai}
\geometry{left=0.7in, right=0.7in, top=1in, bottom=1in}

\renewcommand{\baselinestretch}{1.3}

\title{群论第二章作业}
\author{董建宇 ~~ 202328000807038}

\begin{document}

\maketitle
\titleformat{\section}[hang]{\small}{\thesection}{0.8em}{}{}
\titleformat{\subsection}[hang]{\small}{\thesubsection}{0.8em}{}{}

\begin{enumerate}[1.]
\item 设$H_1$和$H_2$是群$G$的两个子群,证明$H_1$和$H_2$的公共元素的集合也构成群$G$的子群。
\begin{tcolorbox}[breakable, colback = black!5!white, colframe = black]
据题意,显然有$H_1 \cap H_2 \subseteq G$。

\textbf{恒元:}因为$H_1$与$H_2$是群$G$的子群,则$H_1$与$H_2$都包含恒元$E$,则$H_1$和$H_2$的公共元素也包含恒元$E$。

\textbf{结合律:}因为$H_1$和$H_2$的公共元素的集合为群$G$的子集,则结合律自然满足。

\textbf{封闭性:}考虑任意两个$H_1$和$H_2$的公共元素$\alpha$和$\beta$,因为$H_1$和$H_2$是群$G$的两个子群,子群满足封闭性,即:
\[
	\alpha\beta \in H_1, \ \ \ \alpha\beta \in H_2.
\]
则有:
\[
	\alpha\beta \in H_1 \cap H_2.
\]
即$H_1$和$H_2$的公共元素的集合满足封闭性。

\textbf{逆元:}考虑任意一个$H_1$和$H_2$的公共元素$\alpha$,在子群$H_1$和$H_2$中存在$\alpha$的逆元$\alpha_1^{-1}$和$\alpha_2^{-1}$。由于逆元具有唯一性,则$\alpha_1^{-1} = \alpha_2^{-1} = \alpha^{-1}\in H_1 \cap H_2$,即任意一个$H_1$和$H_2$的公共元素$\alpha$,都存在逆元$\alpha^{-1}\in H_1 \cap H_2$。

综上所述,$H_1$和$H_2$的公共元素的集合也构成群$G$的子群。
\end{tcolorbox}

\item 证明: 除恒元外,每个元素的阶都是$2$的群一定是阿贝尔群。
\begin{tcolorbox}[breakable, colback = black!5!white, colframe = black]
考虑任意两个除恒元外的元素$a$和$b$,有:
\[
	a^2 = E, \ \ \ b^2 = E, \ \ \ (ab)^2 = abab = E.
\]
则有:
\[
	aabb = E^2 = E = abab.
\]
两侧同时左乘$a^{-1}$,右乘$b^{-1}$,则得到:
\[
	ab = ba.
\]
即任意两个除恒元外的元素可逆,即除恒元外,每个元素的阶都是2的群一定是阿贝尔群。
\end{tcolorbox}

\item 泡利矩阵定义如下:
\[
	\sigma_1 = \left( \begin{matrix}
		0 & 1 \\
		1 & 0
	\end{matrix} \right), \ \
	\sigma_2 = \left( \begin{matrix}
		0 & -i \\
		i & 0
	\end{matrix} \right), \ \
	\sigma_3 = \left( \begin{matrix}
		1 & 0 \\
		0 & -1
	\end{matrix} \right), \ \ \
\]
\[
	\sigma_a\sigma_b = \delta_{ab}I + i\sum_{d=1}^3 \varepsilon_{abd} \sigma_d \ \ \text{如}\sigma_a^2 = I, \ \ \sigma_1\sigma_2 = i\sigma_3.
\]

其中$\varepsilon_{abd}$是三阶完全反对称张量。证明由$\sigma_1$和$\sigma_2$的所有可能乘积和幂次的集合构成群,列出此群的乘法表,指出此群的阶数,各元素的阶数,群所包含的类和不变子群,不变子群的商群与什么群同构,建立同构关系,证明此群和正方形对称群$D_4$同构。
\begin{tcolorbox}[breakable, colback = black!5!white, colframe = black]
可以计算,$\sigma_1$和$\sigma_2$的所有可能乘积和幂次的集合为:
\[
	\{ \sigma_1, \ -\sigma_1, \ \sigma_2, \ -\sigma_2, \ i\sigma_3, \ -i \sigma_3, \ I, \ -I \}.
\]
可以计算乘法表如下:
\begin{table}[H]
\centering
\begin{tabular}{|c|c|c|c|c|c|c|c|c|}
	\hline
	{} & $I$ & $-I$ & $\sigma_1$ & $-\sigma_1$ & $\sigma_2$ & $-\sigma_2$ & $i\sigma_3$ & $-i\sigma_3$ \\ \hline
	$I$ & $I$ & $-I$ & $\sigma_1$ & $-\sigma_1$ & $\sigma_2$ & $-\sigma_2$ & $i\sigma_3$ & $-i\sigma_3$ \\ \hline
	$-I$ & $-I$ & $I$ & $-\sigma_1$ & $\sigma_1$ & $-\sigma_2$ & $\sigma_2$ & $-i\sigma_3$ & $i\sigma_3$ \\ \hline
	$\sigma_1$ & $\sigma_1$ & $-\sigma_1$ & $I$ & $-I$ & $i\sigma_3$ & $-i\sigma_3$ & $\sigma_2$ & $-\sigma_2$ \\ \hline
	$-\sigma_1$ & $-\sigma_1$ & $\sigma_1$ & $-I$ & $I$ & $-i\sigma_3$ & $i\sigma_3$ & $-\sigma_2$ & $\sigma_2$ \\ \hline
	$\sigma_2$ & $\sigma_2$ & $-\sigma_2$ & $-i\sigma_3$ & $i\sigma_3$ & $I$ & $-I$ & $-\sigma_1$ & $\sigma_1$ \\ \hline
	$-\sigma_2$ & $-\sigma_2$ & $\sigma_2$ & $i\sigma_3$ & $-i\sigma_3$ & $-I$ & $I$ & $\sigma_1$ & $-\sigma_1$ \\ \hline
	$i\sigma_3$ & $i\sigma_3$ & $-i\sigma_3$ & $-\sigma_2$ & $\sigma_2$ & $\sigma_1$ & $-\sigma_1$ & $-I$ & $I$ \\ \hline
	$-i\sigma_3$ & $-i\sigma_3$ & $i\sigma_3$ & $\sigma_2$ & $i\sigma_2$ & $-\sigma_1$ & $\sigma_1$ & $I$ & $-I$ \\ \hline
\end{tabular}
\end{table}
从乘法表中容易看出,该集合满足封闭性与结合律,且存在恒元$I$,且每一个元素存在唯一逆元,即该集合构成群。

群阶数为$8$。

其中$I$的阶数为$1$;$-I, \sigma_1, -\sigma_1, \sigma_2, -\sigma_2$的阶数为$2$;$i\sigma_3, -i\sigma_3$的阶数为$4$。

群包含的类为:
\[
	\{ I \}; \ \ \{ -I \}; \ \ \{ \sigma_1, -\sigma_1 \}; \ \ \{ \sigma_2, -\sigma_2 \}; \ \ \{ i\sigma_3, -i\sigma_3 \}
\]
不变子群为:
\[
	\{ I, -I \}; \ \ \{ I, -I, \sigma_1, -\sigma_1 \}; \ \ \{ I, -I, \sigma_2, -\sigma_2 \}; \ \ \{ I, -I, i\sigma_3, -i\sigma_3 \}.
\]
第一个不变子群的商群同构与四阶反演群$V_4$。映射关系为$\{I, -I\} \to E; \ \{\sigma_1, -\sigma_1\} \to A; \ \{\sigma_2, -\sigma_2\} \to B; \ \{i\sigma_3, -i\sigma_3\} \to C = AB = BA.$

后三个不变子群的商群为二阶群,同构与$C_2$群。映射关系为不变子群映射到恒元$E$,陪集映射到二阶群非恒元元素$A$。

$D_4$群元素与题中群元素对应关系如下:
\setlength{\tabcolsep}{5mm}{
\begin{table}[H]
\centering
\begin{tabular}{|c|c|c|c|c|c|c|c|c|}
	\hline
	$G$ & $I$ & $i\sigma_3$ & $-I$ & $-i\sigma_3$ & $\sigma_1$ & $-\sigma_2$ & $-\sigma_1$ & $\sigma_2$ \\
	\hline
	$D_4$ & $E$ & $T$ & $T^2$ & $T^3$ & $S_1$ & $S_2$ & $S_3$ & $S_4$ \\
	\hline
\end{tabular}
\end{table} }
\end{tcolorbox}

\item 准确到同构,证明九阶群$G$只有两种:循环群$C_9$和直乘群。
\begin{tcolorbox}[breakable, colback = black!5!white, colframe = black]
在九阶群中,除恒元外群元素的阶只能为$3$或$9$。

1. 若至少存在$1$个元素的阶为$9$,那么根据群乘法的封闭性,可知群$G$为九阶循环群$C_9$。

2. 若群$G$中不存在阶为$9$的群元素,且至少存在一个阶为$3$的元素记作$R$,构成$3$阶循环群$C_3 = \{E, R, R^2\}$。考虑其有陪集$C_3A = \{A, B, C\}$,其中$A$为群$G$中元素,且$A\neq E, A\neq R, A\neq R^2$。$B = RA, C = R^2 A$。因为$A,B,C$均为$3$阶群元,则有$A^2, B^2, C^2$既不能等于$E,R,R^2$也不能等于$A, B, C$。即群$G$的群元素可以写为:
\[
	G = \{E, R, R^2, A, B, C, A^2, B^2, C^2\}.
\]
注意到$BA = RA^2 \neq A^2 \neq B^2$,则有$BA = RA^2 = C^2 = R^2AR^2A$。两侧左乘$R^{-1}$,右乘$A^{-1}R$可得:
\[
	AR = RA.
\]
则有$B^2 = RARA = R^2A^2$,综上,群$G$可以写为:
\[
	G = \{ E, R, R^2, A, RA, R^2A, A^2, R^2A^2, RA^2\} = \{E,R,R^2\} \otimes \{E,A,A^2\}.
\]
综上所述,九阶群$G$只有两种:循环群$C_9$和直乘群。
\end{tcolorbox}

\item 设有限群$G$的阶为$g$,$C_a$是群$G$的一个类,含$n(a)$个元素,$S_j$和$S_k$是类$C_a$中任意两个元素,证明群$G$中满足$S_j = PS_kP^{-1}$的元素$P$的数目等于$m(a) = g/n(a)$。
\begin{tcolorbox}[breakable, colback = black!5!white, colframe = black]
对于给定的元素$S_j \in C_\alpha$,设群$G$中所有与$S_j$对易的元素$R$的数目为$m(\alpha)$。可以证明,元素$R$组成的集合$H$构成群$G$的子群。首先,若$R$和$R'$都与$S_j$对易,显然有$RR'$与$S_j$对易,即满足乘法\textbf{封闭性}。\textbf{结合律}显然满足。\textbf{恒元}$E$显然与$S_j$对易。$R$的\textbf{逆元}$R^{-1}$也与$S_j$对易。即$H$为群$G$的子群。

考虑群$G$中不属于子群$H$的元素$T$,且满足
\[
	TS_j T^{-1} = S_i \in C_\alpha.
\]
则子群$H$的左陪集$TH$中任意元素$TR$满足
\[
	TR S_j R^{-1} T^{-1} = T S_j T^{-1} = S_i.
\]
则满足$PS_jP^{-1} = S_i$的元素$P$,有:
\[
	(P^{-1} T) S_j (P^{-1} T)^{-1} = P^{-1} S_i P = S_j.
\]
则有$P^{-1}T \in H$,即$P$属于左陪集$TH$中。因此子群$H$的左陪集$TH$与$S_j$的共轭元素$S_i$具有一一对应关系,即子群$H$的指数等于$S_j$所属类$C_\alpha$中元素数目,即
\[
	n(\alpha) = \frac{g}{m(\alpha)}.
\]
即有:
\[
	m(\alpha) = \frac{g}{n(\alpha)}.
\]
\end{tcolorbox}

\item 以$T$群的子群$C_3 = \{ E, \ R_1, \ R_1^2 \}$为基础,将$C_3$群的乘法表扩充,计算$T$群的乘法表。

\begin{tcolorbox}[breakable, colback = black!5!white, colframe = black]
\begin{enumerate}[(1)]
\item 选取$T_x^2, \ T_y^2, \ T_z^2$做左陪集,第一行$T_x^2C_3$、第二行$T_y^2C_3$、第三行$T_z^2C_3$;

\item 右陪集由左陪集取逆元得到,第四行$C_3T_x^2$、第五行$C_3T_y^2$、第六行$C_3T_z^2$;
\begin{align*}
	(T_x^2C_3)^{-1} =& C_3^{-1} {T_x^2}^{-1} = C_3 T_x^2; \\ 
	(T_y^2C_3)^{-1} =& C_3^{-1} {T_y^2}^{-1} = C_3 T_y^2; \\
	(T_z^2C_3)^{-1} =& C_3^{-1} {T_z^2}^{-1} = C_3 T_z^2.
\end{align*}

\item 左陪集分别右乘$T_x^2, \ T_y^2, \ T_z^2$得到第七至十五行。

\end{enumerate}

\begin{table}[H]
\centering
%\setlength{\tabcolsep}{5mm}{
\begin{tabular}{|c|c|c|c|c|}
	\hline
	左乘 & \makebox[0.12\textwidth][c]{$E$} & \makebox[0.12\textwidth][c]{$R_1$} & \makebox[0.12\textwidth][c]{$R_1^2$} & 右乘 \\ 
	\hline
	$T_x^2$ & $T_x^2$ & $R_4$ & $R_3^2$ &  \\
	\hline 
	$T_y^2$ & $T_y^2$ & $R_3$ & $R_2^2$ &  \\
	\hline 
	$T_z^2$ & $T_z^2$ & $R_2$ & $R_4^2$ &  \\
	\hline
	 & $T_x^2$ & $R_3$ & $R_4^2$ & $T_x^2$ \\
	\hline
	 & $T_y^2$ & $R_2$ & $R_3^2$ & $T_y^2$ \\
	\hline
	 & $T_z^2$ & $R_4$ & $R_2^2$ & $T_z^2$ \\
	\hline
	$T_x^2$ & $E$ & $R_2$ & $R_2^2$ & $T_x^2$ \\
	\hline 
	$T_x^2$ & $T_z^2$ & $R_3$ & $R_1^2$ & $T_y^2$ \\
	\hline 
	$T_x^2$ & $T_y^2$ & $R_1$ & $R_4^2$ & $T_z^2$ \\
	\hline 
	$T_y^2$ & $T_z^2$ & $R_1$ & $R_3^2$ & $T_x^2$ \\
	\hline 
	$T_y^2$ & $E$ & $R_4$ & $R_4^2$ & $T_y^2$ \\
	\hline 
	$T_y^2$ & $T_x^2$ & $R_2$ & $R_1^2$ & $T_z^2$ \\
	\hline 
	$T_z^2$ & $T_y^2$ & $R_4$ & $R_1^2$ & $T_x^2$ \\
	\hline 
	$T_z^2$ & $T_x^2$ & $R_1$ & $R_2^2$ & $T_y^2$ \\
	\hline 
	$T_z^2$ & $E$ & $R_3$ & $R_3^2$ & $T_z^2$ \\
	\hline 
\end{tabular}%}
\end{table}
随后可以进行如下操作:
\begin{enumerate}[(1)]
\item 则我们可以把$T$群乘法表分成四行四列$16$个小方块,$C_3$的乘法表在第一行第一列;

\item 陪集表中第一、二、三行的元素分别替换子群乘法表中的元素,填在第一列的第二、三、四小方块;

\item 陪集表中第四、五、六行的元素分别替换子群乘法表中的元素,填在第一行的第二、三、四小方块;

\item 陪集表中第七、八、九行的元素分别替换子群乘法表中的元素,填在第二行的第二、三、四小方块;

\item 陪集表中第十、十一、十二行的元素分别替换子群乘法表中的元素,填在第三行的第二、三、四小方块;

\item 陪集表中第十三、十四、十五行的元素分别替换子群乘法表中的元素,填在第四行的第二、三、四小方块。

\begin{table}[H]
\centering
\begin{tabular}{|c|c|c|c|c|}
	\hline
	\makebox[0.15\textwidth][c]{} & \makebox[0.15\textwidth][c]{$C_3$} & \makebox[0.15\textwidth][c]{$C_3T_x^2$} & \makebox[0.15\textwidth][c]{$C_3T_y^2$} & \makebox[0.15\textwidth][c]{$C_3T_z^2$} \\
	\hline 
	$C_3$ & $C_3$ & $C_3T_x^2$ & $C_3T_y^2$ & $C_3T_z^2$ \\
	\hline 
	$T_x^2C_3$ & $T_x^2C_3$ & $T_x^2C_3T_x^2$ & $T_x^2C_3T_y^2$ & $T_x^2C_3T_z^2$ \\
	\hline 
	$T_y^2C_3$ & $T_y^2C_3$ & $T_y^2C_3T_x^2$ & $T_y^2C_3T_y^2$ & $T_y^2C_3T_z^2$ \\
	\hline 
	$T_z^2C_3$ & $T_z^2C_3$ & $T_z^2C_3T_x^2$ & $T_z^2C_3T_y^2$ & $T_z^2C_3T_z^2$ \\
	\hline
\end{tabular}
\end{table}
则可以写出$T$群乘法表如下:
\begin{table}[H]
\centering
\begin{tabular}{|c|ccc|ccc|ccc|ccc|}
	\hline
	 & $E$ & $R_1$ & $R_1^2$ & $T_x^2$ & $R_3$ & $R_4^2$ & $T_y^2$ & $R_2$ & $R_3^2$ & $T_z^2$ & $R_4$ & $R_2^2$ \\
	\hline
	$E$ & $E$ & $R_1$ & $R_1^2$ & $T_x^2$ & $R_3$ & $R_4^2$ & $T_y^2$ & $R_2$ & $R_3^2$ & $T_z^2$ & $R_4$ & $R_2^2$ \\
	$R_1$ & $R_1$ & $R_1^2$ & $E$ & $R_3$ & $R_4^2$ & $T_x^2$ & $R_2$ & $R_3^2$ & $T_y^2$ & $R_4$ & $R_2^2$ & $T_z^2$ \\
	$R_1^2$ & $R_1^2$ & $E$ & $R_1$ & $R_4^2$ & $T_x^2$ & $R_3$ & $R_3^2$ & $T_y^2$ & $R_2$ & $R_2^2$ & $T_z^2$ & $R_4$ \\
	\hline
	$T_x^2$ & $T_x^2$ & $R_4$ & $R_3^2$ & $E$ & $R_2$ & $R_2^2$ & $T_z^2$ & $R_3$ & $R_1^2$ & $T_y^2$ & $R_1$ & $R_4^2$ \\
	$R_4$ & $R_4$ & $R_3^2$ & $T_x^2$ & $R_2$ & $R_2^2$ & $E$ & $R_3$ & $R_1^2$ & $T_z^2$ & $R_1$ & $R_4^2$ & $T_y^2$ \\
	$R_3^2$ & $R_3^2$ & $T_x^2$ & $R_4$ & $R_2^2$ & $E$ & $R_2$ & $R_1^2$ & $T_z^2$ & $R_3$ & $R_4^2$ & $T_y^2$ & $R_1$ \\
	\hline 
	$T_y^2$ & $T_y^2$ & $R_3$ & $R_2^2$ & $T_z^2$ & $R_1$ & $R_3^2$ & $E$ & $R_4$ & $R_4^2$ & $T_x^2$ & $R_2$ & $R_1^2$ \\
	$R_3$ & $R_3$ & $R_2^2$ & $T_y^2$ & $R_1$ & $R_3^2$ & $T_z^2$ & $R_4$ & $R_4^2$ & $E$ & $R_2$ & $R_1^2$ & $T_x^2$ \\
	$R_2^2$ & $R_2^2$ & $T_y^2$ & $R_3$ & $R_3^2$ & $T_z^2$ & $R_1$ & $R_4^2$ & $E$ & $R_4$ & $R_1^2$ & $T_x^2$ & $R_2$ \\
	\hline 
	$T_z^2$ & $T_z^2$ & $R_2$ & $R_4^2$ & $T_y^2$ & $R_4$ & $R_1^2$ & $T_x^2$ & $R_1$ & $R_2^2$ & $E$ & $R_3$ & $R_3^2$ \\
	$R_2$ & $R_2$ & $R_4^2$ & $T_z^2$ & $R_4$ & $R_1^2$ & $T_y^2$ & $R_1$ & $R_2^2$ & $T_x^2$ & $R_3$ & $R_3^2$ & $E$ \\
	$R_4^2$ & $R_4^2$ & $T_z^2$ & $R_2$ & $R_1^2$ & $T_y^2$ & $R_4$ & $R_2^2$ & $T_x^2$ & $R_1$ & $R_3^2$ & $E$ & $R_3$ \\
	\hline
\end{tabular}
\end{table}
\end{enumerate}
\end{tcolorbox}

\newpage

\item 群$G$由12个元素组成,它的乘法表如下:
\begin{table}[H]
\centering
\begin{tabular}{c|cccc|c|cccc|c|cccc}
	 & E & A & B & C && D & F & I & J && K & L & M & N \\
	\hline
	E & E & A & B & C && D & F & I & J && K & L & M & N \\ 
	A & A & E & F & I && J & B & C & D && M & N & K & L \\
	B & B & F & A & K && L & E & M & N && I & J & C & D \\
	C & C & I & L & A && K & N & E & M && J & F & D & B \\
	&&&&&&&&&&&&&& \\
	D & D & J & K & L && A & M & N & E && F & I & B & C \\
	F & F & B & E & M && N & A & K & L && C & D & I & J \\
	I & I & C & N & E && M & L & A & K && D & B & J & F \\
	J & J & D & M & N && E & K & L & A && B & C & F & I \\
	&&&&&&&&&&&&&& \\
	K & K & M & J & F && I & D & B & C && N & E & L & A \\
	L & L & N & I & J && F & C & D & B && E & M & A & K \\
	M & M & K & D & B && C & J & F & I && L & A & N & E \\
	N & N & L & C & D && B & I & J & F && A & K & E & M \\
\end{tabular}
\end{table}

(1) 找出群$G$各元素的逆元;

(2) 指出哪些元素可与群中任意元素乘积对易;

(3) 列出各元素的周期和阶;

(4) 找出群$G$各类包含的元素;

(5) 找出群$G$包含哪些不变子群,列出他们的陪集,并指出他们的商群与什么群同构;

(6) 判断群$G$是否与正四面体对称群$T$或与正六边形对称群$D_6$同构。
\begin{tcolorbox}[breakable, colback = black!5!white, colframe = black]
\begin{enumerate}[(1)]

\item 群$G$各元素及其逆元如下:
\begin{table}[H]
\centering
\begin{tabular}{|c|c|c|c|c|c|c|c|c|c|c|c|c|}
	\hline 
	$R\in G$ & E & A & B & C & D & F & I & J & K & L & M & N \\
	\hline
	$R^{-1}$ & E & A & F & I & J & B & C & D & L & K & N & M \\
	\hline
\end{tabular}
\end{table}

\item 群中$E, A$两个元素可与任意元素乘积对易。

\item 群元素$E$周期为$\{E\}$,阶为$1$; \\
群元素$A$周期为$\{A, A^2=E\}$,阶为$2$; \\
群元素$B$周期为$\{B, B^2=A, B^3=F, B^4=E\}$,阶为$4$;\\
群元素$C$周期为$\{C, C^2=A, C^3=I, C^4=E\}$,阶为$4$;\\
群元素$D$周期为$\{D, D^2=A, D^3=J, D^4=E\}$,阶为$4$;\\
群元素$F$周期为$\{F, F^2=A, F^3=B, F^4=E\}$,阶为$4$;\\
群元素$I$周期为$\{I, I^2=A, I^3=C, I^4=E\}$,阶为$4$;\\
群元素$J$周期为$\{J, J^2=A, J^3=D, J^4=E\}$,阶为$4$;\\
群元素$K$周期为$\{K, K^2=N, K^3=A, K^4=M, K^5=L, K^6=E\}$,阶为$6$;\\
群元素$L$周期为$\{L, L^2=M, L^3=A, L^4=N, L^5=K, L^6=E\}$,阶为$6$;\\
群元素$M$周期为$\{M, M^2=N, M^3=E\}$,阶为$3$;\\
群元素$N$周期为$\{N, N^2=M, N^3=E\}$,阶为$3$。

\item 恒元自成一类$\{E\}$;群元素$A$自成一类$\{A\}$,利用乘法表可以得到群$G$的共轭类为:
\[
	\{E\}, \ \{A\}, \ \{B, C, D\}, \ \{F, I, J\}, \ \{K, L\}, \ \{M, N\}.
\]

\item 不变子群$\{E, A\}$,陪集$\{B, F\}, \ \{C, I\}, \ \{D, J\}, \ \{K, M\}, \ \{L, N\}$,其商群同构与$D_3$群。\\
不变子群$\{E, M, N\}$,陪集$\{A, K, L\}, \ \{B, C, D\}, \ \{F, I, J\}$,其商群同构与$C_4$群。\\
不变子群$\{E, A, K, L, M, N\}$,陪集$\{B, C, D, F, I, J\}$,其商群同构与$C_2$群。

\item $T$群不包含阶数为$6$的元素,则群$G$与群$T$不同构。$D_6$群不包含阶数为$4$的元素,则群$G$与群$T$不同构。
\end{enumerate}
\end{tcolorbox}

\end{enumerate}
\end{document}

\begin{tcolorbox}[breakable, colback = black!5!white, colframe = black]

\end{tcolorbox}