\documentclass[reqno,a4paper,12pt]{amsart}

\usepackage{amsmath,amssymb,amsthm,geometry,xcolor,soul,graphicx}
\usepackage{titlesec}
\usepackage{enumerate}
\usepackage{lipsum}
\usepackage{listings}
\usepackage{ytableau} %杨图
%\RequirePackage[most]{tcolorbox}
\usepackage{float}  %可以在colorbox环境下添加table环境 此时,\begin{table}[H] 位置需要H
\usepackage{braket}
\allowdisplaybreaks[4] %align公式跨页
\usepackage{xeCJK}
\setCJKmainfont[AutoFakeBold = true]{Kai}
\geometry{left=0.7in, right=0.7in, top=1in, bottom=1in}

\renewcommand{\baselinestretch}{1.3}

\title{群论第四章作业}
\author{董建宇 ~~ 202328000807038}

\begin{document}

\maketitle

\titleformat{\section}[hang]{\small}{\thesection}{0.8em}{}{}
\titleformat{\subsection}[hang]{\small}{\thesubsection}{0.8em}{}{}
%\ytableausetup{boxsize=1.5em} %杨表格式

\begin{enumerate}[1.]

\item 把下列置换化为无公共客体的轮换乘积:
\begin{enumerate}[(1)]
\item 
\begin{align*}
	(1 \ 2)(2 \ 3)(1 \ 2) =& \left( \begin{aligned}
		&1 & &2 & 3 \\
		&2 & &1 & 3
	\end{aligned}\right)\left( \begin{aligned}
		&1 & &2 & 3 \\
		&1 & &3 & 2
	\end{aligned}\right)\left( \begin{aligned}
		&1 & &2 & 3 \\
		&2 & &1 & 3
	\end{aligned}\right) \\
	=& \left( \begin{aligned}
		&3 & &1 & 2 \\
		&3 & &2 & 1
	\end{aligned}\right)\left( \begin{aligned}
		&2 & &1 & 3 \\
		&3 & &1 & 2
	\end{aligned}\right)\left( \begin{aligned}
		&1 & &2 & 3 \\
		&2 & &1 & 3
	\end{aligned}\right) = \left( \begin{aligned}
		&1 & &2 & 3 \\
		&3 & &2 & 1
	\end{aligned}\right) = (1 \ 3).
\end{align*}

\item 
\begin{align*}
	(1 \ 2 \ 3)(1 \ 3 \ 4)(3 \ 2 \ 1) =& \left( \begin{aligned}
		&1 & &2 & &3 & 4 \\
		&2 & &3 & &1 & 4
	\end{aligned}\right)\left( \begin{aligned}
		&1 & &2 & &3 & 4 \\
		&3 & &2 & &4 & 1
	\end{aligned}\right)\left( \begin{aligned}
		&3 & &2 & &1 & 4 \\
		&2 & &1 & &3 & 4
	\end{aligned}\right) \\
	=& \left( \begin{aligned}
		&2 & &3 & &4 & 1 \\
		&3 & &1 & &4 & 2
	\end{aligned}\right)\left( \begin{aligned}
		&2 & &1 & &3 & 4 \\
		&2 & &3 & &4 & 1
	\end{aligned}\right)\left( \begin{aligned}
		&3 & &2 & &1 & 4 \\
		&2 & &1 & &3 & 4
	\end{aligned}\right) \\
	=& \left( \begin{aligned}
		&3 & &2 & &1 & 4 \\
		&3 & &1 & &4 & 2
	\end{aligned}\right) = (1 \ 4 \ 2).
\end{align*}

\item 
\begin{align*}
	(1 \ 2 \ 3 \ 4)^{-1} = \left( \begin{aligned}
		&1 & &2 & &3 & 4 \\
		&2 & &3 & &4 & 1
	\end{aligned}\right)^{-1} = \left( \begin{aligned}
		&2 & &3 & &4 & 1 \\
		&1 & &2 & &3 & 4
	\end{aligned}\right) = (4 \ 3 \ 2 \ 1).
\end{align*}

\item 
\begin{align*}
	(1 \ 2 \ 4 \ 5)(4 \ 3 \ 2 \ 6) =& (5 \ 1 \ 2)(2 \ 4)(4 \ 3 \ 2)(2 \ 6) = (5 \ 1 \ 2)(4 \ 2)(2 \ 4 \ 3)(2 \ 6) \\
	=&(5 \ 1 \ 2)(4 \ 3)(2 \ 6) = (5 \ 1 \ 2 \ 6)(4 \ 3).
\end{align*}

\item 
\begin{align*}
	(1 \ 2 \ 3)(4 \ 2 \ 6)(3 \ 4 \ 5 \ 6) =& (1 \ 2 \ 3)(2 \ 6 \ 4)(6 \ 3 \ 4)(4 \ 5) = (1 \ 2 \ 3)(2 \ 6 \ 4)(4 \ 6 \ 3)(4 \ 5) \\
	=& (1 \ 2 \ 3)(2 \ 6 \ 3)(4 \ 5) = (1 \ 2 \ 3)(3 \ 2 \ 6)(4 \ 5) = (1 \ 2 \ 6)(4 \ 5).
\end{align*}

\end{enumerate}

\item 写出对应下列杨表的杨算符
\begin{enumerate}[(1)]
\item 
\begin{ytableau}
1 & 2 & 3 \\
4 \\
\end{ytableau}

\[
	[E + (1 \ 2) + (1 \ 3) + (2 \ 3) + (1 \ 2 \ 3) + (1 \ 3 \ 2)][E - (1 \ 4)].
\]

\item 
\begin{ytableau}
1 & 2 \\
3 & 4 \\
\end{ytableau}

\[
	[E + (1 \ 2)] [E + (3 \ 4)] [E - (1 \ 3)] [E - (2 \ 4)].
\]

\item 
\begin{ytableau}
1 & 2 & 3 & 4 \\
5 \\
\end{ytableau}

\begin{align*}
	&[E + (1 \ 2) + (1 \ 3) + (1 \ 4) + (2 \ 3) + (2 \ 4) + (3 \ 4) + (1 \ 2 \ 3) + (1 \ 3 \ 2) + (1 \ 2 \ 4) + (1 \ 4 \ 2)\\
	+& (1 \ 3 \ 4) + (1 \ 4 \ 3) + (2 \ 3 \ 4) + (2 \ 4 \ 3) + (1 \ 2 \ 3 \ 4) + (1 \ 2 \ 4 \ 3) + (1 \ 3 \ 2 \ 4) + (1 \ 3 \ 4 \ 2) \\
	+& (1 \ 4 \ 2 \ 3) + (1 \ 4 \ 3 \ 2) + (1 \ 2)(3 \ 4) + (1 \ 3)(2 \ 4) + (1 \ 4)(2 \ 3)][E - (1 \ 5)].
\end{align*}

\end{enumerate}

\item 具体写出$S_4$群恒元按杨算符的展开式。
\begin{proof}
恒元可以写为
\[
	E = \frac{A}{24} + \frac{B}{12} + \frac{C}{8}.
\]
其中:
\begin{align*}
	A =& \begin{ytableau}
		1 & 2 & 3 & 4 
	\end{ytableau} + 
	\begin{ytableau}
		1 \\
		2 \\
		3 \\
		4
	\end{ytableau} 
	= 2[E + (1 \ 2)(3 \ 4) + (1 \ 3)(2 \ 4) + (1 \ 4)(2 \ 3) + (1 \ 2 \ 3) + (1 \ 3 \ 2) \\
	&+ (1 \ 2 \ 4) + (1 \ 4 \ 2) + (1 \ 3 \ 4) + (1 \ 4 \ 3) + (2 \ 3 \ 4) + (2 \ 4 \ 3)]; \\
	B =& \begin{ytableau}
		1 & 2 \\
		3 & 4
	\end{ytableau} + 
	\begin{ytableau}
		1 & 3 \\
		2 & 4
	\end{ytableau}
	= E + (1 \ 2) + (3 \ 4) + (1 \ 2)(3 \ 4) - (1 \ 3) - (2 \ 1 \ 3) - (4 \ 3 \ 1) \\
	&- (2 \ 1 \ 4 \ 3) - (2 \ 4) - (1 \ 2 \ 4) - (3 \ 4 \ 2) - (1 \ 2 \ 3 \ 4) + (1 \ 3)(2 \ 4) + (1 \ 3 \ 2 \ 4) \\
	&+ (3 \ 1 \ 4 \ 2) + (1 \ 4)(3 \ 2) + E + (1 \ 3) + (2 \ 4) + (1 \ 3)(2 \ 4) - (1 \ 2) - (3 \ 1 \ 2) \\
	&- (4 \ 2 \ 1) - (3 \ 1 \ 4 \ 2) - (3 \ 4) - (1 \ 3 \ 4) - (2 \ 4 \ 3) - (1 \ 3 \ 2 \ 4) + (1 \ 2)(3 \ 4) \\
	&+ (1 \ 2 \ 3 \ 4) + (2 \ 1 \ 4 \ 3) + (1 \ 4)(2 \ 3); \\
	C =& \begin{ytableau}
		1 & 2 & 3 \\
		4 \\
	\end{ytableau} + 
	\begin{ytableau}
		1 & 4 \\
		2 \\
		3
	\end{ytableau} + 
	\begin{ytableau}
		1 & 2 & 4 \\
		3 \\
	\end{ytableau} + 
	\begin{ytableau}
		1 & 3 \\
		2 \\
		4
	\end{ytableau} + 
	\begin{ytableau}
		1 & 3 & 4 \\
		2 \\
	\end{ytableau} + 
	\begin{ytableau}
		1 & 2 \\
		3 \\
		4
	\end{ytableau} \\
	=& 6E - 2(2 \ 3)(1 \ 4) - 2(2 \ 4)(1 \ 3) - 2(4 \ 3)(1 \ 2).
\end{align*}
\end{proof}

\item 下列两正则杨算符乘积$\mathcal{Y}_1\mathcal{Y}_2$不为零,$R$把正则杨表$\mathcal{Y}_2$变成$\mathcal{Y}_1$,试把$R$表示成属于杨表$\mathcal{Y}_2$的横向置换$P_2$和纵向置换$Q_2$的乘积$P_2Q_2$,再表示成属于杨表$\mathcal{Y}_1$的横向置换$P_1$和纵向置换$Q_1$的乘积$P_1Q_1$.
\[
	\mathcal{Y}_1 = \begin{ytableau}
		1 & 2 & 3 & 4 \\
		5 & 6 & 7 \\
		8 & 9
	\end{ytableau}, \ \ 
	\mathcal{Y}_2 = \begin{ytableau}
		1 & 2 & 4 & 7 \\
		3 & 6 & 9 \\
		6 & 8
	\end{ytableau}.
\]

\begin{proof}
置换为:
\begin{align*}
	R =& \left( \begin{aligned}
		&1 && 2 && 4 && 7 && 3 && 5 && 9 && 6 && 8 \\
		&1 && 2 && 3 && 4 && 5 && 6 && 7 && 8 && 9
	\end{aligned} \right) = (4 \ 3 \ 5 \ 6 \ 8 \ 9 \ 7) \\
	=& (7 \ 4) (4 \ 3 \ 5 \ 6 \ 8 \ 9) = (7 \ 4) (3 \ 5) (5 \ 6 \ 8 \ 9 \ 4) \\
	=& (7 \ 4) (3 \ 5) (5\ 6 \ 8 \ 9) (9 \ 4) = (7 \ 4) (3 \ 5) (9 \ 5) (5 \ 6 \ 8) (9 \ 4) \\
	=& (7 \ 4) (3 \ 5 \ 9) (6 \ 8) (8 \ 5) (9 \ 4) = P_1Q_1.
\end{align*}

其中
\[
	P_1 = (7 \ 4) (3 \ 5 \ 9) (6 \ 8); \ \ Q_1 = (8 \ 5) (9 \ 4).
\]

或者
\begin{align*}
	R =& (4 \ 3 \ 5 \ 6 \ 8 \ 9 \ 7) \\
	=& (4 \ 3) (3 \ 5 \ 6 \ 8 \ 9 \ 7) = (4 \ 3)(5 \ 6)(6 \ 8 \ 9 \ 7)(7 \ 3) \\
	=& (4 \ 3)(5 \ 6)(7 \ 6)(6 \ 8 \ 9)(7 \ 3) = (4 \ 3)(5 \ 6 \ 7)(8 \ 9)(9 \ 6)(7 \ 3).
\end{align*}

其中
\[
	P_2 = (4 \ 3)(5 \ 6 \ 7)(8 \ 9); \ \ Q_2 = (9 \ 6)(7 \ 3).
\]
\end{proof}

\item 用列表法计算$S_5$群生成元$(1 \ 2)$和$(1 \ 2 \ 3 \ 4 \ 5)$在不可约表示$[2,2,1]$中的表示矩阵。

\begin{proof}
正交杨算符分别为$\mathcal{Y}_1[E - (2 \ 5)], \ \mathcal{Y}_2, \ \mathcal{Y}_3, \ \mathcal{Y}_4, \ \mathcal{Y}_5$,列表法计算表示矩阵如下:
\begin{table}[H]
\begin{tabular}{|l|l|l|l|l|l|}
\hline
(1 \ 2) & \begin{ytableau}
2 & 1 \\
3 & 4 \\
5
\end{ytableau} & \begin{ytableau}
2 & 1 \\
3 & 5 \\
4
\end{ytableau} & \begin{ytableau}
2 & 3 \\
1 & 4 \\
5
\end{ytableau} & \begin{ytableau}
2 & 3 \\
1 & 5 \\
4
\end{ytableau} & \begin{ytableau}
2 & 4 \\
1 & 5 \\
3
\end{ytableau} \\ \hline
\begin{ytableau}
1 & 2 \\
3 & 4 \\
5
\end{ytableau}-\begin{ytableau}
1 & 5 \\
3 & 4 \\
2
\end{ytableau} & 1-0 & 0-0 & 1-0 & 0-0 & 0+1 \\ \hline
\begin{ytableau}
1 & 2 \\
3 & 5 \\
4
\end{ytableau} & 0 & 1 & 0 & -1 & 1 \\ \hline
\begin{ytableau}
1 & 3 \\
2 & 4 \\
5
\end{ytableau} & 0 & 0 & -1 & 0 & 0 \\ \hline
\begin{ytableau}
1 & 3 \\
2 & 5 \\
4
\end{ytableau} & 0 & 0 & 0 & -1 & 0 \\ \hline
\begin{ytableau}
1 & 4 \\
2 & 5 \\
3
\end{ytableau} & 0 & 0 & 0 & 0 & -1 \\ \hline
\end{tabular}
\end{table}
\end{proof}

\item 用等效方法计算$S_6$群各类在下列不可约表示中的特征标。

(1) 表示$[3,2,1]$; (2) 表示$[3,3]$; (3) 表示$[2,2,2]$
\begin{proof}
\
\begin{table}[H]
\begin{tabular}{|c|c|c|c|}
\hline
\text{类} & [3,2,1] & [3,3] & [2,2,2] \\ \hline
$(1^6)$ & 16 & 5 & 5 \\ \hline
$(2,1^4)$ & 0 & 1 & -1 \\ \hline
$(2^2,1^2)$ & 0 & 1 & 1 \\ \hline
$(2^3)$ & 0 & -3 & 3 \\ \hline
$(3,1^3)$ & -2 & -1 & -1 \\ \hline
$(3,2,1)$ & 0 & 1 & -1 \\ \hline
$(3^2)$ & -2 & 2 & 2 \\ \hline
$(4,1^2)$ & 0 & -1 & 1 \\ \hline
$(4,2)$ & 0 & -1 & -1 \\ \hline
$(5,1)$ & 1 & 0 & 0 \\ \hline
$(6)$ & 0 & 0 & 0 \\ \hline
\end{tabular}
\end{table}
\end{proof}


\item 分别写出$S_6$群相邻客体对换$P_a$在不可约表示$[3,3]$和$[2,2,2]$正交基中的实正交表示矩阵形式。因为下式两边的表示是等价的
\[
	[2,2,2] \simeq [1^6]\times[3,3].
\]
试计算它们间的相似变换矩阵$X$。
\begin{proof}
对应杨图$[2,2,2]$的正则杨表为:
\[
	\begin{ytableau}
		1 & 2 \\
		3 & 4 \\
		5 & 6
	\end{ytableau} \ \ 
	\begin{ytableau}
		1 & 2 \\
		3 & 5 \\
		4 & 6
	\end{ytableau} \ \ 
	\begin{ytableau}
		1 & 3 \\
		2 & 4 \\
		5 & 6
	\end{ytableau} \ \ 
	\begin{ytableau}
		1 & 3 \\
		2 & 5 \\
		4 & 6
	\end{ytableau} \ \ 
	\begin{ytableau}
		1 & 4 \\
		2 & 5 \\
		3 & 6
	\end{ytableau};
\]

杨图$[3,3]$的正则杨表为:
\[
	\begin{ytableau}
		1 & 2 & 3 \\
		4 & 5 & 6
	\end{ytableau} \ \ 
	\begin{ytableau}
		1 & 2 & 4 \\
		3 & 5 & 6
	\end{ytableau} \ \ 
	\begin{ytableau}
		1 & 2 & 5 \\
		3 & 4 & 6
	\end{ytableau} \ \ 
	\begin{ytableau}
		1 & 3 & 4 \\
		2 & 5 & 6
	\end{ytableau} \ \ 
	\begin{ytableau}
		1 & 3 & 5 \\
		2 & 4 & 6
	\end{ytableau}.
\]

相邻客体对换$P_a$在表示$[2,2,2]$和$[1^6]\times [3,3]$中的表示矩阵分别为:
\begin{table}[H]
\begin{tabular}{|c|c|c|}
\hline
 & $[2,2,2]$ & $[1^6]\times[3,3]$ \\ \hline
$P_1$ & $\left(\begin{matrix}
	1 & 0 & 0 & 0 & 0 \\
	0 & 1 & 0 & 0 & 0 \\
	0 & 0 & -1 & 0 & 0 \\
	0 & 0 & 0 & -1 & 0 \\
	0 & 0 & 0 & 0 & -1 \\
\end{matrix}
\right)$ & $\left(\begin{matrix}
	-1 & 0 & 0 & 0 & 0 \\
	0 & -1 & 0 & 0 & 0 \\
	0 & 0 & -1 & 0 & 0 \\
	0 & 0 & 0 & 1 & 0 \\
	0 & 0 & 0 & 0 & 1 \\
\end{matrix}
\right)$ \\ \hline
$P_2$ & $\frac{1}{2}\left(\begin{matrix}
	-1 & 0 & \sqrt{3} & 0 & 0 \\
	0 & -1 & 0 & \sqrt{3} & 0 \\
	\sqrt{3} & 0 & 1 & 0 & 0 \\
	0 & \sqrt{3} & 0 & 1 & 0 \\
	0 & 0 & 0 & 0 & -2 \\
\end{matrix}
\right)$ & $\frac{1}{2}\left(\begin{matrix}
	-2 & 0 & 0 & 0 & 0 \\
	0 & 1 & 0 & -\sqrt{3} & 0 \\
	0 & 0 & 1 & 0 & -\sqrt{3} \\
	0 & -\sqrt{3} & 0 & -1 & 0 \\
	0 & 0 & -\sqrt{3} & 0 & -1 \\
\end{matrix}
\right)$ \\ \hline
$P_3$ & $\frac{1}{3}\left(\begin{matrix}
	3 & 0 & 0 & 0 & 0 \\
	0 & -3 & 0 & 0 & 0 \\
	0 & 0 & -3 & 0 & 0 \\
	0 & 0 & 0 & -1 & \sqrt{8} \\
	0 & 0 & 0 & \sqrt{8} & 1 \\
\end{matrix}
\right)$ & $\frac{1}{3}\left(\begin{matrix}
	1 & -\sqrt{8} & 0 & 0 & 0 \\
	-\sqrt{8} & -1 & 0 & 0 & 0 \\
	0 & 0 & -3 & 0 & 0 \\
	0 & 0 & 0 & -3 & 0 \\
	0 & 0 & 0 & 0 & 3 \\
\end{matrix}
\right)$ \\ \hline
$P_4$ & $\frac{1}{2}\left(\begin{matrix}
	-1 & \sqrt{3} & 0 & 0 & 0 \\
	\sqrt{3} & 1 & 0 & 0 & 0 \\
	0 & 0 & -1 & \sqrt{3} & 0 \\
	0 & 0 & \sqrt{3} & 1 & 0 \\
	0 & 0 & 0 & 0 & -2 \\
\end{matrix}
\right)$ & $\frac{1}{2}\left(\begin{matrix}
	-2 & 0 & 0 & 0 & 0 \\
	0 & 1 & -\sqrt{3} & 0 & 0 \\
	0 & -\sqrt{3} & -1 & 0 & 0 \\
	0 & 0 & 0 & 1 & -\sqrt{3} \\
	0 & 0 & 0 & -\sqrt{3} & -1 \\
\end{matrix}
\right)$ \\ \hline
$P_5$ & $\left(\begin{matrix}
	1 & 0 & 0 & 0 & 0 \\
	0 & -1 & 0 & 0 & 0 \\
	0 & 0 & 1 & 0 & 0 \\
	0 & 0 & 0 & -1 & 0 \\
	0 & 0 & 0 & 0 & -1 \\
\end{matrix}
\right)$ & $\left(\begin{matrix}
	-1 & 0 & 0 & 0 & 0 \\
	0 & -1 & 0 & 0 & 0 \\
	0 & 0 & 1 & 0 & 0 \\
	0 & 0 & 0 & -1 & 0 \\
	0 & 0 & 0 & 0 & 1 \\
\end{matrix}
\right)$ \\ \hline
\end{tabular}
\end{table}

两组矩阵中,对应矩阵可以通过两次变换得到,先把非对角元改号,再关于反对角线做转置。

注意到$[2,2,2]$非零对角元只在$12,13,24,34,45$出现,只改变第一行、第四行符号就可以全部变号,则$X$矩阵可以写作$X = YZ$,$Y$对角元第1、4行为$-1$,其余为$1$,$Z$的反对角元全为$1$,则
\[
	X = \left( \begin{matrix}
		0 & 0 & 0 & 0 & -1 \\
		0 & 0 & 0 & 1 & 0 \\
		0 & 0 & 1 & 0 & 0 \\
		0 & -1 & 0 & 1 & 0 \\
		1 & 0 & 0 & 0 & 0
	\end{matrix} \right)
\]

满足
\[
	X^{-1} [2,2,2] X = [1^6]\times [3,3].
\]
\end{proof}

\item 用立特伍德-理查森规则计算下列置换群表示外积的约化
\begin{enumerate}[(1)]
\item $[3,2,1] \otimes [3]$, 
\begin{proof}
\begin{align*}
	\begin{ytableau}
		\ & \ & \ \\
		\ & \ \\
		\ 
	\end{ytableau} \times 
	\begin{ytableau}
		1 & 1 & 1
	\end{ytableau} \ \simeq \ &
	\begin{ytableau}
		\ & \ & \ & 1 & 1 & 1 \\
		\ & \ \\
		\ 
	\end{ytableau} \ \oplus
	\begin{ytableau}
		\ & \ & \ & 1 & 1 \\
		\ & \ & 1 \\
		\ 
	\end{ytableau} \\ 
	&\oplus
	\begin{ytableau}
		\ & \ & \ & 1 & 1 \\
		\ & \ \\
		\ & 1
	\end{ytableau} \ \oplus
	\begin{ytableau}
		\ & \ & \ & 1 & 1 \\
		\ & \ \\
		\ \\
		1
	\end{ytableau} \\
	&\oplus
	\begin{ytableau}
		\ & \ & \ & 1 \\
		\ & \ & 1 \\
		\ & 1
	\end{ytableau} \ \oplus 
	\begin{ytableau}
		\ & \ & \ & 1 \\
		\ & \ & 1 \\
		\ \\
		1
	\end{ytableau} \\
	&\oplus
	\begin{ytableau}
		\ & \ & \ & 1 \\
		\ & \ \\
		\ & 1 \\
		1
	\end{ytableau} \ \oplus
	\begin{ytableau}
		\ & \ & \ \\
		\ & \ & 1 \\
		\ & 1 \\
		1
	\end{ytableau}. \\
	[3,2,1]\times[3] \simeq& [6,2,1] \oplus [5,3,1] \oplus [5,2,2] \oplus [5,2,1,1] \oplus [4,3,2] \\
	&\oplus[4,3,1,1] \oplus[4,2,2,1] \oplus[3,3,2,1].
\end{align*}

维数验证
\[
	\frac{9!}{6!3!}\times16\times1 =1344=105+162+120+189+168+216+216+168.
\]
\end{proof}

\item $[3,2] \otimes [2,1]$, 
\begin{proof}
\begin{align*}
	\begin{ytableau}
		\ & \ & \ \\
		\ & \
	\end{ytableau} \times
	\begin{ytableau}
		1 & 1 \\
		2
	\end{ytableau} \ \simeq \ &
	\begin{ytableau}
		\ & \ & \ & 1 & 1 \\
		\ & \ & 2 
	\end{ytableau} \oplus
	\begin{ytableau}
		\ & \ & \ & 1 & 1 \\
		\ & \ \\
		2
	\end{ytableau} \\
	&\oplus
	\begin{ytableau}
		\ & \ & \ & 1 \\
		\ & \ & 1 & 2
	\end{ytableau} \oplus
	\begin{ytableau}
		\ & \ & \ & 1 \\
		\ & \ & 1 \\
		2
	\end{ytableau} \\
	&\oplus
	\begin{ytableau}
		\ & \ & \ & 1 \\
		\ & \ & 2 \\
		1
	\end{ytableau} \oplus
	\begin{ytableau}
		\ & \ & \ & 1 \\
		\ & \ \\
		1 & 2
	\end{ytableau} \\
	&\oplus
	\begin{ytableau}
		\ & \ & \ & 1 \\
		\ & \ \\
		1 \\
		2
	\end{ytableau} \oplus
	\begin{ytableau}
		\ & \ & \ \\
		\ & \ & 1 \\
		1 & 2
	\end{ytableau} \\
	& \oplus
	\begin{ytableau}
		\ & \ & \ \\
		\ & \ & 1 \\
		1 \\
		2
	\end{ytableau} \oplus
	\begin{ytableau}
		\ & \ & \ \\
		\ & \ \\
		1 & 1 \\
		2
	\end{ytableau}. \\
	[3,2] \times [2,1] \simeq & [5,3] \oplus [5,2,1] \oplus [4,4] \oplus 2[4,3,1] \oplus [4,2,2] \\
	&\oplus [4,2,1,1] \oplus [3,3,2] \oplus [3,3,1,1] \oplus [3,2,2,1].
\end{align*}

维数验证
\[
	\frac{8!}{5!3!}\times5\times2 = 560 = 28+64+14+2\times70+56+90+42+56+70.
\]
\end{proof}

\item $[2,1] \otimes [4,2^3]$.
\begin{proof}
\begin{align*}
	\begin{ytableau}
		1 & 1 \\
		2
	\end{ytableau} \times 
	\begin{ytableau}
		\ & \ & \ & \ \\
		\ & \ \\
		\ & \ \\
		\ & \ 
	\end{ytableau} \ \simeq \ &
	\begin{ytableau}
		\ & \ & \ & \ & 1 & 1 \\
		\ & \ & 2 \\
		\ & \ \\
		\ & \ 
	\end{ytableau} \oplus
	\begin{ytableau}
		\ & \ & \ & \ & 1 & 1 \\
		\ & \ \\
		\ & \ \\
		\ & \ \\
		2
	\end{ytableau} \\
	&\oplus 
	\begin{ytableau}
		\ & \ & \ & \ & 1 \\
		\ & \ & 1 & 2 \\
		\ & \ \\
		\ & \ 
	\end{ytableau} \oplus
	\begin{ytableau}
		\ & \ & \ & \ & 1 \\
		\ & \ & 1 \\
		\ & \ & 2 \\
		\ & \ 
	\end{ytableau} \\
	&\oplus
	\begin{ytableau}
		\ & \ & \ & \ & 1 \\
		\ & \ & 1 \\
		\ & \ \\
		\ & \ \\
		2
	\end{ytableau} \oplus
	\begin{ytableau}
		\ & \ & \ & \ & 1 \\
		\ & \ & 2 \\
		\ & \ \\
		\ & \ \\
		1
	\end{ytableau} \\
	&\oplus
	\begin{ytableau}
		\ & \ & \ & \ & 1 \\
		\ & \ \\
		\ & \ \\
		\ & \ \\
		1 & 2
	\end{ytableau} \oplus 
	\begin{ytableau}
		\ & \ & \ & \ & 1 \\
		\ & \ \\
		\ & \ \\
		\ & \ \\
		1 \\
		2
	\end{ytableau} \\
	& \oplus
	\begin{ytableau}
		\ & \ & \ & \ \\
		\ & \ & 1 & 1 \\
		\ & \ & 2 \\
		\ & \ 
	\end{ytableau} \oplus
	\begin{ytableau}
		\ & \ & \ & \ \\
		\ & \ & 1 & 1 \\
		\ & \ \\
		\ & \ \\
		2
	\end{ytableau} \\
	&\oplus
	\begin{ytableau}
		\ & \ & \ & \ \\
		\ & \ & 1 \\
		\ & \ & 2 \\
		\ & \ \\
		1
	\end{ytableau} \oplus
	\begin{ytableau}
		\ & \ & \ & \ \\
		\ & \ & 1 \\
		\ & \ \\
		\ & \ \\
		1 & 2
	\end{ytableau} \\
	&\oplus
	\begin{ytableau}
		\ & \ & \ & \ \\
		\ & \ & 1 \\
		\ & \ \\
		\ & \ \\
		1 \\
		2
	\end{ytableau} \oplus
	\begin{ytableau}
		\ & \ & \ & \ \\
		\ & \ \\
		\ & \ \\
		\ & \ \\
		1 & 1 \\
		2
	\end{ytableau}. \\
	[2,1]\times[4,2,2,2] \simeq& [6,3,2,2] \oplus [6,2,2,2,1] \oplus [5,4,2,2] \oplus [5,3,3,2] \\
	&\oplus 2[5,3,2,2,1] \oplus [5,2,2,2,2] \oplus [5,2,2,2,1,1] \oplus [4,4,3,2] \\
	&\oplus [4,4,2,2,1] \oplus [4,3,3,2,1] \oplus [4,3,2,2,2] \\
	&\oplus [4,3,2,2,1,1] \oplus [4,2,2,2,2,1].
\end{align*}

维数验证
\begin{align*}
	\frac{13!}{3!10!}\times2\times300 =& 171600 = 12012 + 9009 + 12870 + 11583 + 2\times 21450 + 5005 \\
	& + 10296 + 8580 + 12870 + 15015 + 8580 + 17160 + 5720.
\end{align*}
\end{proof}

\end{enumerate}

\item 用立特伍德-查利森规则计算,$S_6$群下列不可约表示关于子群$S_3\otimes S_3$的分导表示,按子群不可约表示的约化
\begin{enumerate}[(1)]
\item $[4,2]$,
\begin{proof}
\[
	\begin{ytableau}
		\ & \ & \ & 1 \\
		1 & 1 
	\end{ytableau} \  \oplus
	\begin{ytableau}
		\ & \ & \ & 1 \\
		1 & 2 
	\end{ytableau} \ \oplus
	\begin{ytableau}
		\ & \ & 1 & 1 \\
		\ & 1 
	\end{ytableau} \ \oplus
	\begin{ytableau}
		\ & \ & 1 & 1 \\
		\ & 2 
	\end{ytableau}.
\]

\[
	[4,2] \simeq [3]\times[3] \oplus [3] \times [2,1] \oplus [2,1] \times [3] \oplus [2,1]\times[2,1].
\]

维数验证:
\[
	9 = 1\times 1 + 1\times 2 + 2\times 1 + 2\times 2.
\]
\end{proof}

\item $[2,2,1,1]$,
\begin{proof}
\[
	\begin{ytableau}
		\ & \ \\
		\ & 1 \\
		1 \\
		2
	\end{ytableau} \ \oplus
	\begin{ytableau}
		\ & \ \\
		\ & 1 \\
		2 \\
		3
	\end{ytableau} \ \oplus
	\begin{ytableau}
		\ & 1 \\
		\ & 2 \\
		\ \\
		1
	\end{ytableau} \ \oplus
	\begin{ytableau}
		\ & 1 \\
		\ & 2 \\
		\ \\
		3
	\end{ytableau}.
\]
\[
	[2,2,1,1] \simeq [2,1] \times [2,1] \oplus [2,1] \times [1,1,1] \oplus [1,1,1] \times [2,1] \oplus [1,1,1] \times [1,1,1].
\]

维数验证
\[
	 9 = 2 \times 2 + 2 \times 1 + 1 \times 2 + 1 \times 1.
\]
\end{proof}

\item $[3,3]$.
\begin{proof}
\[
	\begin{ytableau}
		\ & \ & \ \\
		1 & 1 & 1 
	\end{ytableau} \ \oplus
	\begin{ytableau}
		\ & \ & 1 \\
		\ & 1 & 2 
	\end{ytableau}.
\]
\[
	[3,3] \simeq [3] \times [3] \oplus [2,1] \times [2,1]
\]

维数验证
\[
	5 = 1 \times 1 + 2 \times 2.
\]
\end{proof}

\end{enumerate}


\end{enumerate}

\end{document}

%杨图例子
\ytableausetup{boxsize=1.5em}
\begin{ytableau}
1 & 2 & 3 \\
4 & 5 \\
6
\end{ytableau}