\documentclass[reqno,a4paper,12pt]{amsart}

\usepackage{amsmath,amssymb,amsthm,geometry,xcolor,soul,graphicx}
\usepackage{titlesec}
\usepackage{enumerate}
\usepackage{lipsum}
\usepackage{listings}
\RequirePackage[most]{tcolorbox}
\usepackage{braket}
\allowdisplaybreaks[4] %align公式跨页
\usepackage{xeCJK}
\setCJKmainfont{Kai}
\geometry{left=0.7in, right=0.7in, top=1in, bottom=1in}

\renewcommand{\baselinestretch}{1.3}

\title{群论
作业}
\author{董建宇 ~~ 202328000807038}

\begin{document}

\maketitle
\titleformat{\section}[hang]{\small}{\thesection}{0.8em}{}{}
\titleformat{\subsection}[hang]{\small}{\thesubsection}{0.8em}{}{}

\textbf{1.}找到相似变换矩阵$M$使
\[
	M^{-1}\left(
	\begin{matrix}
		0 & -\cos\theta & \sin\theta\sin\varphi \\
		\cos\theta & 0 & -\sin\theta\cos\varphi \\
		-\sin\theta\sin\varphi & \sin\theta\cos\varphi & 0
	\end{matrix}
	\right)M = \left(
	\begin{matrix}
		0 & -1 & 0 \\
		1 & 0 & 0 \\
		0 & 0 & 0
	\end{matrix}
	\right)
\]
\begin{tcolorbox}[breakable, colback = black!5!white, colframe = black]
由于矩阵$M$可逆,等式两侧左乘$M$,并将矩阵$M$按列分块,即$M = (\vec{m}_1 ~~ \vec{m}_2 ~~ \vec{m}_3)$。则题目等式化为:
\begin{align*}
	\left(
	\begin{matrix}
		0 & -\cos\theta & \sin\theta\sin\varphi \\
		\cos\theta & 0 & -\sin\theta\cos\varphi \\
		-\sin\theta\sin\varphi & \sin\theta\cos\varphi & 0
	\end{matrix}
	\right)
	\left(
	\begin{matrix}
		\vec{m}_1 & \vec{m}_2 & \vec{m}_3
	\end{matrix}
	\right)
	=&
	\left(
	\begin{matrix}
		\vec{m}_1 & \vec{m}_2 & \vec{m}_3
	\end{matrix}
	\right)
	\left(
	\begin{matrix}
		0 & -1 & 0 \\
		1 & 0 & 0 \\
		0 & 0 & 0
	\end{matrix}
	\right) \\
	=&
	\left(
	\begin{matrix}
		\vec{m}_2 & -\vec{m}_1 & 0
	\end{matrix}
	\right)
\end{align*}
对于每个列向量,记作:$\vec{m}_i = (a_i ~~ b_i ~~ c_i)^T$。考虑$\vec{m}_3$有:
\[
	\left(
	\begin{matrix}
		0 & -\cos\theta & \sin\theta\sin\varphi \\
		\cos\theta & 0 & -\sin\theta\cos\varphi \\
		-\sin\theta\sin\varphi & \sin\theta\cos\varphi & 0
	\end{matrix}
	\right)
	\left(
	\begin{matrix}
		a_3 \\
		b_3 \\ 
		c_3
	\end{matrix}
	\right)
	=
	\left(
	\begin{matrix}
		0 \\ 
		0 \\ 
		0
	\end{matrix}
	\right)
\]
可得:
\begin{align*}
	a_3=& \tan\theta\cos\varphi ~ c_3; \\
	b_3=& \tan\theta\sin\varphi ~ c_3.
\end{align*}
考虑归一化$\vert a_3 \vert^2 + \vert b_3 \vert^2 + \vert c_3 \vert^2 = 1$,并选取$c_3$的相位因子为$0$,可得:
\[
	\vec{m}_3 = (
	\begin{matrix}
		\sin\theta\cos\varphi & \sin\theta\sin\varphi & \cos\theta
	\end{matrix}
	)^T.
\]
考虑$\vec{m}_1,\vec{m}_2$,有:
\begin{align*}
	\left(
	\begin{matrix}
		0 & -\cos\theta & \sin\theta\sin\varphi \\
		\cos\theta & 0 & -\sin\theta\cos\varphi \\
		-\sin\theta\sin\varphi & \sin\theta\cos\varphi & 0
	\end{matrix}
	\right)
	\left(
	\begin{matrix}
		a_1 \\ 
		b_1 \\ 
		c_1
	\end{matrix}
	\right)
	=& 
	\left(
	\begin{matrix}
		a_2 \\ 
		b_2 \\ 
		c_2
	\end{matrix}
	\right) \\
	\left(
	\begin{matrix}
		0 & -\cos\theta & \sin\theta\sin\varphi \\
		\cos\theta & 0 & -\sin\theta\cos\varphi \\
		-\sin\theta\sin\varphi & \sin\theta\cos\varphi & 0
	\end{matrix}
	\right)
	\left(
	\begin{matrix}
		a_2 \\ 
		b_2 \\ 
		c_2
	\end{matrix}
	\right)
	=& 
	-\left(
	\begin{matrix}
		a_1 \\
		b_1 \\
		c_1
	\end{matrix}
	\right)
\end{align*}
即
\[
	\left(
	\begin{matrix}
		0 & -\cos\theta & \sin\theta\sin\varphi \\
		\cos\theta & 0 & -\sin\theta\cos\varphi \\
		-\sin\theta\sin\varphi & \sin\theta\cos\varphi & 0
	\end{matrix}
	\right)^2
	\left(
	\begin{matrix}
		a_j \\ 
		b_j \\ 
		c_j
	\end{matrix}
	\right)
	=
	-\left(
	\begin{matrix}
		a_j \\ 
		b_j \\ 
		c_j
	\end{matrix}
	\right)
\]
其中,$j=1,2$。则有:
\begin{align*}
	\vec{m}_1 =& (
	\begin{matrix}
		\sin\varphi & -\cos\varphi & 0
	\end{matrix}
	)^T; \\
	\vec{m}_2 =& (
	\begin{matrix}
		\cos\theta\cos\varphi & \cos\theta\sin\varphi & -\sin\theta
	\end{matrix}
	)^T.
\end{align*}
综上,矩阵$M$可为:
\[
	M = \left(
	\begin{matrix}
		\sin\varphi & \cos\theta\cos\varphi & \sin\theta\cos\varphi \\
		-\cos\varphi & \cos\theta\sin\varphi & \sin\theta\sin\varphi \\
		0 & -\sin\theta & \cos\theta
	\end{matrix}
	\right)
\]
可以计算$M$的行列式为:
\[
	\mathbf{det}(M) = 1 \neq 0.
\]
即矩阵$M$可逆,即矩阵$M$为题目所求相似变换矩阵。
\end{tcolorbox}

\textbf{2.}设$R=\left( \begin{matrix} 1 & 0 \\ 0 & -1 \end{matrix} \right)$,$S=\cfrac{1}{2}\left( \begin{matrix} -1 & -\sqrt{3} \\ \sqrt{3} & -1 \end{matrix} \right)$,找到相似变换矩阵$X$使:
\[
	X^{-1}(R\otimes R)X = \left(
	\begin{matrix}
		1 & 0 & 0 & 0 \\
		0 & -1 & 0 & 0 \\
		0 & 0 & 1 & 0 \\
		0 & 0 & 0 & -1
	\end{matrix}
	\right),
	X^{-1}(S\otimes S)X = \frac{1}{2}\left(
	\begin{matrix}
		2 & 0 & 0 & 0 \\
		0 & 2 & 0 & 0 \\
		0 & 0 & -1 & -\sqrt{3} \\
		0 & 0 & \sqrt{3} & -1
	\end{matrix}
	\right)
\]

\begin{tcolorbox}[breakable, colback = black!5!white, colframe = black]
记$A = R\otimes R,~B = S\otimes S$,可以计算矩阵$A,B$分别为:
\[
	A = \left( \begin{matrix}
		1 & 0 & 0 & 0 \\
		0 & -1 & 0 & 0 \\
		0 & 0 & -1 & 0 \\
		0 & 0 & 0 & 1
	\end{matrix} \right), ~ 
	B = \frac{1}{4} \left( \begin{matrix}
		1 & \sqrt{3} & \sqrt{3} & 3 \\
		-\sqrt{3} & 1 & -3 & \sqrt{3} \\
		-\sqrt{3} & -3 & 1 & \sqrt{3} \\
		3 & -\sqrt{3} & -\sqrt{3} & 1
	\end{matrix} \right).
\]
将矩阵$X$按列向量分块,即:
\[
	X = \left( \begin{matrix}
		\vec{x}_1 & \vec{x}_2 & \vec{x}_3 & \vec{x}_4
	\end{matrix} \right).
\]
则矩阵$X$满足:
\begin{align*}
	A\left( \begin{matrix}
		\vec{x}_1 & \vec{x}_2 & \vec{x}_3 & \vec{x}_4
	\end{matrix} \right) =& X \left( \begin{matrix}
		1 & 0 & 0 & 0 \\
		0 & -1 & 0 & 0 \\
		0 & 0 & 1 & 0 \\
		0 & 0 & 0 & -1
	\end{matrix} \right) = \left( \begin{matrix}
		\vec{x}_1 & -\vec{x}_2 & \vec{x}_3 & -\vec{x}_4
	\end{matrix} \right); \\
	B\left( \begin{matrix}
		\vec{x}_1 & \vec{x}_2 & \vec{x}_3 & \vec{x}_4
	\end{matrix} \right) =& X \left( \begin{matrix}
		1 & 0 & 0 & 0 \\
		0 & 1 & 0 & 0 \\
		0 & 0 & -\frac{1}{2} & -\frac{\sqrt{3}}{2} \\
		0 & 0 & \frac{\sqrt{3}}{2} & -\frac{1}{2}
	\end{matrix} \right) = \left( \begin{matrix}
		\vec{x}_1 & \vec{x}_2 & -\frac{1}{2}\vec{x}_3-\frac{\sqrt{3}}{2}\vec{x}_4 & \frac{\sqrt{3}}{2}\vec{x}_3-\frac{1}{2}\vec{x}_4
	\end{matrix} \right).
\end{align*}
令$\vec{x}_i = \left( \begin{matrix}
	a_i & b_i & c_i & d_i
\end{matrix} \right)^T$,则根据$x_1,x_3$是$A$本征值为$1$的本征向量,$x_2,x_4$是本征值为$-1$的本征向量,可以得知:
\[
	b_1 = c_1 = b_3 = c_3 = 0; ~ a_2 = d_2 = a_4 = d_4 = 0.
\]
则有:
\begin{align*}
	\frac{1}{4} \left( \begin{matrix}
		1 & \sqrt{3} & \sqrt{3} & 3 \\
		-\sqrt{3} & 1 & -3 & \sqrt{3} \\
		-\sqrt{3} & -3 & 1 & \sqrt{3} \\
		3 & -\sqrt{3} & -\sqrt{3} & 1
	\end{matrix} \right) \left( \begin{matrix}
		a_1 \\
		0 \\
		0 \\
		d_1
	\end{matrix} \right) &= \left( \begin{matrix}
		a_1 \\
		0 \\
		0 \\
		d_1
	\end{matrix} \right); \\
	{}
	\frac{1}{4} \left( \begin{matrix}
		1 & \sqrt{3} & \sqrt{3} & 3 \\
		-\sqrt{3} & 1 & -3 & \sqrt{3} \\
		-\sqrt{3} & -3 & 1 & \sqrt{3} \\
		3 & -\sqrt{3} & -\sqrt{3} & 1
	\end{matrix} \right) \left( \begin{matrix}
		0 \\
		b_2 \\
		c_2 \\
		0
	\end{matrix} \right) &= \left( \begin{matrix}
		0 \\
		b_2 \\
		c_2 \\
		0
	\end{matrix} \right); \\
	{}
	\frac{1}{4} \left( \begin{matrix}
		1 & \sqrt{3} & \sqrt{3} & 3 \\
		-\sqrt{3} & 1 & -3 & \sqrt{3} \\
		-\sqrt{3} & -3 & 1 & \sqrt{3} \\
		3 & -\sqrt{3} & -\sqrt{3} & 1
	\end{matrix} \right) \left( \begin{matrix}
		a_3 \\
		0 \\
		0 \\
		d_3
	\end{matrix} \right) &= \left( \begin{matrix}
		-\frac{1}{2}a_3 \\
		-\frac{\sqrt{3}}{2}b_4 \\
		-\frac{\sqrt{3}}{2}c_4 \\
		-\frac{1}{2}d_4
	\end{matrix} \right).
\end{align*}
可以解得:
\begin{align*}
	\vec{x}_1 =& \left( \begin{matrix}
		\frac{1}{\sqrt{2}} & 0 & 0 & \frac{1}{\sqrt{2}}
	\end{matrix} \right)^T; \\
	{}
	\vec{x}_2 =& \left( \begin{matrix}
		0 & \frac{1}{\sqrt{2}} & -\frac{1}{\sqrt{2}} & 0
	\end{matrix} \right)^T; \\
	\vec{x}_3 =& \left( \begin{matrix}
		\frac{1}{\sqrt{2}} & 0 & 0 & -\frac{1}{\sqrt{2}}
	\end{matrix} \right)^T; \\
	\vec{x}_4 =& \left( \begin{matrix}
		0 & \frac{1}{\sqrt{2}} & \frac{1}{\sqrt{2}} & 0
	\end{matrix} \right)^T.
\end{align*}
综上所述,相似变换矩阵$X$为:
\[
	X = \left( \begin{matrix}
		\frac{1}{\sqrt{2}} & 0 & \frac{1}{\sqrt{2}} & 0 \\
		0 & \frac{1}{\sqrt{2}} & 0 & \frac{1}{\sqrt{2}} \\
		0 & -\frac{1}{\sqrt{2}} & 0 & \frac{1}{\sqrt{2}} \\
		\frac{1}{\sqrt{2}} & 0 & -\frac{1}{\sqrt{2}} & 0 \\
	\end{matrix} \right)
\]
\end{tcolorbox}

\end{document}